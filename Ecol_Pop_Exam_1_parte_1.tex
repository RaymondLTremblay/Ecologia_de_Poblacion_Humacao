% Options for packages loaded elsewhere
\PassOptionsToPackage{unicode}{hyperref}
\PassOptionsToPackage{hyphens}{url}
%
\documentclass[
]{article}
\usepackage{amsmath,amssymb}
\usepackage{lmodern}
\usepackage{iftex}
\ifPDFTeX
  \usepackage[T1]{fontenc}
  \usepackage[utf8]{inputenc}
  \usepackage{textcomp} % provide euro and other symbols
\else % if luatex or xetex
  \usepackage{unicode-math}
  \defaultfontfeatures{Scale=MatchLowercase}
  \defaultfontfeatures[\rmfamily]{Ligatures=TeX,Scale=1}
\fi
% Use upquote if available, for straight quotes in verbatim environments
\IfFileExists{upquote.sty}{\usepackage{upquote}}{}
\IfFileExists{microtype.sty}{% use microtype if available
  \usepackage[]{microtype}
  \UseMicrotypeSet[protrusion]{basicmath} % disable protrusion for tt fonts
}{}
\makeatletter
\@ifundefined{KOMAClassName}{% if non-KOMA class
  \IfFileExists{parskip.sty}{%
    \usepackage{parskip}
  }{% else
    \setlength{\parindent}{0pt}
    \setlength{\parskip}{6pt plus 2pt minus 1pt}}
}{% if KOMA class
  \KOMAoptions{parskip=half}}
\makeatother
\usepackage{xcolor}
\usepackage[margin=1in]{geometry}
\usepackage{graphicx}
\makeatletter
\def\maxwidth{\ifdim\Gin@nat@width>\linewidth\linewidth\else\Gin@nat@width\fi}
\def\maxheight{\ifdim\Gin@nat@height>\textheight\textheight\else\Gin@nat@height\fi}
\makeatother
% Scale images if necessary, so that they will not overflow the page
% margins by default, and it is still possible to overwrite the defaults
% using explicit options in \includegraphics[width, height, ...]{}
\setkeys{Gin}{width=\maxwidth,height=\maxheight,keepaspectratio}
% Set default figure placement to htbp
\makeatletter
\def\fps@figure{htbp}
\makeatother
\setlength{\emergencystretch}{3em} % prevent overfull lines
\providecommand{\tightlist}{%
  \setlength{\itemsep}{0pt}\setlength{\parskip}{0pt}}
\setcounter{secnumdepth}{-\maxdimen} % remove section numbering
\ifLuaTeX
  \usepackage{selnolig}  % disable illegal ligatures
\fi
\IfFileExists{bookmark.sty}{\usepackage{bookmark}}{\usepackage{hyperref}}
\IfFileExists{xurl.sty}{\usepackage{xurl}}{} % add URL line breaks if available
\urlstyle{same} % disable monospaced font for URLs
\hypersetup{
  pdftitle={Ecol\_Pop\_Exam\_1\_parte\_1},
  pdfauthor={BIOL4558},
  hidelinks,
  pdfcreator={LaTeX via pandoc}}

\title{Ecol\_Pop\_Exam\_1\_parte\_1}
\author{BIOL4558}
\date{Agosto 2021}

\begin{document}
\maketitle

{
\setcounter{tocdepth}{2}
\tableofcontents
}
\hypertarget{las-preguntas-del-examen-de-esta-parte-proviene-del-capiyulo-1}{%
\subsection{Las preguntas del examen de esta parte proviene del capiyulo
1:}\label{las-preguntas-del-examen-de-esta-parte-proviene-del-capiyulo-1}}

\begin{enumerate}
\def\labelenumi{\arabic{enumi}.}
\item
  In 1993, when the first edition of this book was written, the world's
  human population was expected to double in size in approximately 50
  years. Assuming population growth is continuous, calculate r for the
  human population.
\item
  If the population size in 1993 was 5.4 billion, what was the projected
  population size for the year 2021 using the a growth rate with
  doubling time of 50 years?
\item
  The future is here!. On October 17, 2021 the best estimate of the
  world population size was 7.796 billion - different than that
  projected by the model in 1993. To find out the current estimate of
  the world populaion size, visit the website maintained by the US
  Census Bureau.
\end{enumerate}

\href{https://www.census.gov/popclock/}{Census Population size website}

\begin{enumerate}
\def\labelenumi{\arabic{enumi}.}
\setcounter{enumi}{3}
\item
  What is today's date and how large is the population of the world
  today. What is the difference in nnu mber of humans on the planet as
  compared to that predicted?
\item
  Considering the present population size what is the observed growth
  rate of human population on the planet \textbf{per year}, use the
  population size of 5.538 billion for 1993.
\item
  You are studying a population of beetles of size 3000. During one
  month period, you record 400 births and 150 deaths in this population.
  Estimate r and project the population size in from month 1 to 6
  months.
\item
  For five consecutive days you count the size of a growing population
  of flatworms as 100, 158, 315, 298, 794. Plot the logarithm (base e)
  of population size to estimate r.
\item
  A population of annual grasses increases in size by 12\% every year.
  This species was introduced as cow foder. What is the approximate
  doubling time?
\end{enumerate}

\end{document}
